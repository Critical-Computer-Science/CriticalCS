% AUTHOR NAME HERE
\documentclass[11pt]{article}
\usepackage[utf8]{inputenc}
\usepackage{listings}
\usepackage{upquote,textcomp}
\usepackage{amsmath,amsfonts,amsthm}
\usepackage{url}
\usepackage{graphicx}
\graphicspath{ {./images/} }
\usepackage{fullpage}
\usepackage{hyperref}

% Hard figure placement
\usepackage{float}

\usepackage{color}

\usepackage[coloroftodonotes]{todonotes}

\definecolor{mygreen}{rgb}{0,0.6,0}
\definecolor{mygray}{rgb}{0.5,0.5,0.5}
\definecolor{mymauve}{rgb}{0.58,0,0.82}

\newcommand{\duedate}[1]{\date{\textbf{Due: #1}}}


\lstset{frame=tb,
  language=,
  aboveskip=3mm,
  belowskip=3mm,
  showstringspaces=false,
  columns=flexible,
  keepspaces=true,
  basicstyle={\small\ttfamily},
  numbers=none,
  numberstyle=\tiny\color{black},
  keywordstyle=\color{black},
  commentstyle=\color{black},
  stringstyle=\color{black},
  breaklines=true,
  breakatwhitespace=true,
  tabsize=3
}

\lstset{frame=tb,
  language=Python,
  aboveskip=3mm,
  belowskip=3mm,
  showstringspaces=false,
  columns=flexible,
  basicstyle={\small\ttfamily},
  numbers=none,
  numberstyle=\tiny\color{mygray},
  keywordstyle=\color{blue},
  commentstyle=\color{mygreen},
  stringstyle=\color{mymauve},
  breaklines=true,
  breakatwhitespace=true,
  tabsize=3
}

\textwidth  6.5in
\oddsidemargin +0.0in
\evensidemargin +0.0in
\textheight 9.0in
\topmargin -0.5in

\setlength{\parindent}{0pt}
\setlength{\parskip}{3pt}


\setcounter{part}{1}

\newenvironment{Part}[2]
{
    \begin{center}
        \Large\textbf{Part \thepart: #1}\\
        \large\textit{#2}
        \stepcounter{part}
    \end{center}
}

\title{\textbf{[REQUIRES $names$ LIBRARY]Teacher Turing Test}}
\author{\textit{Loops, Decisions, Simple String Parsing, Intro to Turing's AI Theory, Abstraction}}
\duedate{\todo{due date}}


\begin{document}
\thispagestyle{empty}

\maketitle

\begin{Part}{Building Q\&A}{Ask Questions to the Given Teachers}

\null\quad\quad This homework will introduce you to Turing's theory of artificial intelligence through a simple version of his most well-known work: the Turing Test. This test is designed as a Q\&A game where one participant tries to determine if the other one is a human or a robot based on their responses. For this homework, we'll be doing a very simple version of this game.\\
\null\quad\quad You're job in this part is to build the Q\&A platform for you Turing Test. We have provided you with three classes: "Human" which will represent the human teachers; "Robot" which will represent the robot teacher; and "District" which will represent the whole district. "Human" and "Robot" objects have the same interface, so you won't be able to know if you're interacting with a Human or a Robot based on the code alone. We'll talk about Districts more in the next part.\bigskip\\
\null\quad You are provided with a list of Districts and a list of questions to ask the Teachers. For this part, you are to loop through each District. In each District, you are to loop through each Teacher. For each Teacher, you are to display their name and ask them each question and display their response. These are the methods at your exposure for Teachers:\\
\begin{itemize}
    \item getFirstName(): Returns a String of the Teacher's first name
    \item getLastName(): Returns a String of the Teacher's last name
    \item ask(question): Takes in a String and returns the Teacher's answer to the question. If the Teacher doesn't have an answer, they'll respond that they don't know
    \item \_\_str\_\_(): This function gets called with str(obj) or print(obj). It returns the Teacher's name and what they are (Human or Robot). ONLY USE THIS AT THE END, NOT FOR THE Q\&A SECTION
\end{itemize}
\null\quad\quad To use any of these functions, take a Teacher object (let's say $t$) and call $t.method(argument)$. To access the teachers for District $d$, loop through $d.teachers$.
\end{Part}
\newpage
\begin{Part}{Turing Test}{Build User Interface and Parse Results}

\null\quad\quad Now to build the test itself! Now that you have the questions and responses, you want your user to guess whether or not the Teacher is a Human or a Robot. How you choose to do this is up to you, but you need to track their responses in a list so we can check them later. You'll notice that there are a LOT of Teachers to go through. Once you feel like your input and output are working, you can automate this process using random.choice if you'd like, but don't delete your input code (simply comment it out).\\
\null\quad\quad Once you've collected your responses, you'll need to calculate the budget for the District you're investigating. In this case, you can simply use a list and assume that the index in the $district$ list will match up with the index of the $budget$ list (e.g. if "Seattle" is index 2, "Seattle"s budget will also be index 2). Here are the rules for budgeting:\\
\begin{itemize}
    \item A Human costs \$25,000 per year
    \item A Robot costs \$15,000 per year
\end{itemize}
\null\quad\quad You will be building this budget based on what the user guessed from the Turing Test, so your final budgets might not be accurate. We'll check which ones were right and which ones were wrong.\\
\null\quad\quad Each District comes pre-loaded with a known budget. For each District, you will print one of the following statements:\\
\begin{itemize}
    \item If the calculated budget matches the known budget: Print "We have enough money to pay everyone a fair wage in [District.name]"
    \item If the calculated budget is less than the known budget: Print "The [District.name] district will be underfunded by \$[difference//1000],000 next year"
    \item If the calculated budget is more than the known budget: Print "We do not have the funds for the [District.name] district budget. We will have a debt of \$[difference//1000],000"
\end{itemize}
\null\quad\quad At the end, if you're curious what the actual budget was or what the Teachers were, you can call $print(d)$ for some District $d$ to see all information about the District.
\end{Part}

\begin{Part}{README}{Final Thoughts}
\null\quad\quad Please respond to the questions in the provided README.txt file and submit it along with your code.
\end{Part}

\end{document}
